\capitulo{7}{Conclusiones y Líneas de trabajo futuras}

Este proyecto me pareció útil e interesante, ya que se centra en el desarrollo web de una aplicación que serviría para facilitar la gestión de las reservas de las aulas. Mi idea inicial era realizarlo en Java ya que habíamos visto parte de ello en la carrera y en las prácticas que realice el primer cuatrimestre. Como estaba adjudicado ya, decidí hacerlo en Python y aprender a trabajar con ello.
\newline
Como conclusiones del desarrollo del proyecto, centrándome en el trabajo que he realizado con la API de Outlook, destaco la ayuda de dicha API para alcanzar los resultados, pero dicha API aún tiene funcionalidades en desarrollo y en versión Beta, por lo que se puede mejorar en este aspecto, como por ejemplo introduciendo la posibilidad de compartir un calendario con otra persona mediante una llamada a la API, esta funcionalidad apareció el 21 de Mayo con el proyecto muy avanzado y para poder hacer uso de ella requiere de invitación manual, ya que se centra en administrar el uso compartido y la delegación de permisos sobre los calendarios ya compartidos. Este aspecto queda como una mejora futura de la aplicación, de forma que desde la propia aplicación se pueda compartir el aula con el propietario que se le asigne, de momento cualquier cambio sobre el propietario del aula hay que asignárselo manualmente desde la cuenta del administrador, que controla todas las aulas, compartiendo el calendario con la persona que sea propietario.\newline
En cuanto a una conclusión que sacar de la interfaz de la aplicación podría destacar que no incluí ninguna representación de calendario para los eventos, por otra parte introduje una filtro de búsqueda, quizás podría ser una mejora visual para la aplicación.


