\capitulo{7}{Conclusiones y Líneas de trabajo futuras}

Este proyecto me pareció útil e interesante, ya que se centra en el desarrollo web de una aplicación que serviría para facilitar la gestión de las reservas de las aulas. Este es un problema que se da en muchos lugares a la hora de reservar y la idea es facilitar la tarea de las reservas y que no ocurran solapamientos entre dichas reservas.
\newline
Una primera conclusión que me gustaría dejar es relacionada al trabajo con Azure, ya que el despliegue se realizó ahí como se ha comentado a lo largo de la documentación. El problema que se encontró al final, respecto al fin de licencia de estudiante y tener que ampliarlo a una suscripción por pago. En este proyecto trabajando con una base de datos y un app service, los servicios hay que arrancarlos antes de poder ejecutarlo en la nube, se puede establecer que no haya pausas, surgiendo nuestro problema, donde se gastó el dinero al facturar como ejecutando continuamente la base de datos. Por lo que sugiero una diferente plataforma o alguna alternativa como quizás no ejecutar las acciones que trabajan con la base de datos en cada ejecución, ya que desde local también trabajaba con esta base de datos subida. Finalmente y tras conversar con el servicio técnico de Azure, se llegó a la conclusión de que para la presentación del proyecto sería mejor un cambio en la configuración de la base de datos , estableciendo un uso básico para que esté abierta continuamente y se pueda probar en cualquier momento.\newline 
Otro impedimento que tuve al utilizar Azure fue la imposición de utilizar SQL Server en vez de MySQL como estaba utilizando al comienzo en local, esto se debe a que en la suscripción de Azure se permite crear gratuitamente bases de datos pero de SQL Server solamente.\newline
Como conclusiones del desarrollo del proyecto, centrándome en el trabajo que se ha realizado con la API de Microsoft, es importante destacar que Microsoft está actualmente implantada en la mayoría de instituciones educativas, como bien podemos obtener las licencias de sus aplicaciones, por lo que a la hora de querer implantarlo dentro de la universidad no se tendrán problemas de licencias y todo estará enlazado más fácilmente.\newline
Sobre esta API REST en concreto cabe destacar su ayuda para alcanzar los resultados, pero dicha API aún tiene funcionalidades en desarrollo y en versión Beta, por lo que se puede mejorar en este aspecto, como por ejemplo introduciendo la posibilidad de compartir un calendario con otra persona mediante una llamada a la API, esta funcionalidad apareció en la documentación con el proyecto muy avanzado y para poder hacer uso de ella requiere de invitación manual, ya que se centra en administrar el uso compartido y la delegación de permisos sobre los calendarios ya compartidos. Este aspecto queda como una mejora futura de la aplicación, de forma que desde la propia aplicación se pueda compartir el aula con el propietario que se le asigne, de momento cualquier cambio sobre el propietario del aula hay que asignárselo manualmente desde la cuenta del administrador, que controla todas las aulas, compartiendo el calendario con la persona que sea propietario.\newline
Otro punto que ralentizó el avance del proyecto fue el problema de los solapamientos, ya que la API puede devolver los eventos existentes, pero para controlar todas las horas que se puedan solapar, no sólo que se muestre como solapamiento la misma hora hubo que recurrir a consultas de la base de datos. Los calendarios de Outlook permiten los solapamientos y no disponen de una función que bloquee esto.\newline
En cuanto al entorno de trabajo ningún problema al trabajar con Visual Studio Code, que dispone de una documentación bien detallada y de muchas extensiones que agilizan el trabajo.\newline
Otra posible mejora futura sería una integración con ubuVirtual, como acabo de mencionar mediante el uso de Microsoft ya existente, para que las reservas aparezcan en el calendario de exámenes si fueran exámenes o como notificaciones de eventos especiales en caso de charlas o talleres.\newline 



