\apendice{Especificación de Requisitos}

\section{Introducción}
En este apartado de la documentación explicaré  los objetivos que se perseguían al comienzo del proyecto, lo que se ha conseguido y los requisitos funcionales y no funcionales.
\section{Objetivos generales}
\begin{itemize}
	\item Desarrollar una aplicación web cliente servidor, en la que los usuarios con permisos necesarios puedan reservas aulas para los profesores.
	\item Ofrecer la posibilidad de reservar un único día, más de uno seguidos o un día a la semana el mismo aula.
	\item Ofrecer al usuario una vista sobre las aulas existentes, si es administrador podrá modificarlas o borrarlas también.
	\item Ofrecer al usuario una vista con las franjas horarias reservadas y libres de cada aula.
	\item Controlar el solapamiento de horas.
\end{itemize}
\section{Catalogo de requisitos}
En esta sección enumeraré los requisitos funcionales de la aplicación y los usuarios que intervienen.
\textbf{Usuario:} será quién use la aplicación, según los privilegios les hay de tres tipos:
    \begin{itemize}
        \item Sin privilegios: Puede consultar todas las aulas y su ocupación.
        \item Propietario: Puede consultar todas las aulas y su ocupación. Además puede realizar reservas sobre las aulas de las que es propietario.
        \item Administrador: Puede consultar todas las aulas y su ocupación. Además puede realizar reservas sobre todas las aulas. Puede modificar la información de las aulas o borrarlas. Se encarga de conceder los privilegios a los usuarios.
    \end{itemize}
\subsection{Requisitos funcionales} 
\begin{itemize}
    \item \textbf{RF.1} Controlar el acceso mediante un inicio de sesión.
    \begin{itemize}
        \item \textbf{RF.1.1} Acceso a la aplicación mediante el login de Outlook.
    \end{itemize}
    \item \textbf{RF.2} La aplicación tiene que poder distinguir entre el tipo de usuario que ha accedido. Distingue tres tipos:
    \begin{itemize}
        \item \textbf{RF.2.1} Diferencia los usuarios sin permisos.
        \item \textbf{RF.2.2} Diferencia los usuarios propietarios de aulas.
        \item \textbf{RF.2.3} Diferencia los usuarios administradores.
    \end{itemize}
    \item \textbf{RF.3} La aplicación tiene que poder mostrar información sobre las aulas.
    \begin{itemize}
        \item \textbf{RF.3.1} Permite ver el listado de aulas.
        \item \textbf{RF.3.2} Permite ver características de cada aula.
    \end{itemize}
    \item \textbf{RF.4} La aplicación tiene que poder modificar las aulas.
    \begin{itemize}
        \item \textbf{RF.4.1} Permite modificar las características de las aulas.
        \item \textbf{RF.4.2} Permite el borrado de aulas.
    \end{itemize}
    \item \textbf{RF.5} La aplicación tiene que poder realizar una reserva de tres tipos.
    \begin{itemize}
        \item \textbf{RF.5.1} Permite reservar un sólo día de una hora a otra especificada.
        \item \textbf{RF.5.2} Permite reservar varios seguidos a la misma hora.
        \item \textbf{RF.5.3} Permite reservar un día a la semana a la misma hora durante el tiempo especificado. 
    \end{itemize}
    \item \textbf{RF.6} La aplicación tiene que poder mostrar todos los eventos creados.
    \begin{itemize}
        \item \textbf{RF.6.1} Permite modificar los eventos.
        \item \textbf{RF.6.2} Permite borrar los eventos.
    \end{itemize}
    \item \textbf{RF.7} La aplicación tiene que poder crear un aula nueva para un manejo posterior.
     
    
    
\end{itemize}


\subsection{Requisitos no funcionales}
\begin{itemize}
	\item \textbf{RNF.1 Usabilidad} Proporcionar una página web sencilla de usar e intuitiva.
	\item \textbf{RNF.2 Disponibilidad} El tiempo para iniciar o reiniciar el sistema no es mayor a 5 minutos.
	\item \textbf{RNF.3 Seguridad} Se controla que no se pueda acceder sin haber iniciado sesión y una vez iniciada sesión, se controla el acceso a las distintas partes de la aplicación.
	\item \textbf{RFN.4 Mantenibilidad} Es sencillo añadir nuevas funcionalidades.
\end{itemize}
\section{Especificación de requisitos}
\subsection{Diagrama de casos de uso}

\imagen{CU_SinPermisos}{Diagrama de casos de uso de un usuario sin permisos.}
\imagen{CU_Propietario}{Diagrama de casos de uso de un usuario propietario.}
\imagen{CU_Administrador}{Diagrama de casos de uso de un usuario administrador.}

\subsection{Especificación de los casos de uso}
\tablaSmallSinColores{Caso de uso 0: Iniciar sesión.}{p{3cm} p{.75cm} p{9.5cm}}{tablaUC0}{
  \multicolumn{3}{l}{Caso de uso 0: Iniciar sesión.} \\
 }
 {
  Descripción                            & \multicolumn{2}{p{10.25cm}}{Permite al usuario iniciar sesión en la aplicación.} \\\hline
  \multirow{6}{3.5cm}{Requisitos}   &\multicolumn{2}{p{10.25cm}}{RF-1} \\\cline{2-3}
                                         & \multicolumn{2}{p{10.25cm}}{RF-1.1} 
                                         \\\cline{2-3}
                                         & \multicolumn{2}{p{10.25cm}}{RF-2}
                                         \\\cline{2-3}
                                         &
                                    \multicolumn{2}{p{10.25cm}}{RF-2.1}
                                    \\\cline{2-3}
                                         &
                                    \multicolumn{2}{p{10.25cm}}{RF-2.2}
                                    \\\cline{2-3}
                                         &
                                    \multicolumn{2}{p{10.25cm}}{RF-2.3}
                                         \\\hline
  Precondiciones                         &  \multicolumn{2}{p{10.25cm}}{Disponer de una cuenta de Outlook.}   \\\hline
  \multirow{2}{3.5cm}{Secuencia normal}  & Paso & Acción \\\cline{2-3}
                                         & 1    & El usuario pincha el botón 'Iniciar sesión'.
  \\\cline{2-3}
                                         & 2    & Se redirige al inicio de sesión de Outlook.
  \\\cline{2-3}
                                         & 3    & Se introducen los datos de inicio de sesión de la cuenta de Outlook.
    \\\cline{2-3}
                                         & 4    & Se inicia la sesión si existe la cuenta.
                                         \\\hline
  Postcondiciones                        & \multicolumn{2}{p{10.25cm}}{Si la sesión se ha iniciado correctamente, se lleva al usuario a la página de inicio. Si no se vuelve a pedir que introduzca una cuenta existente.} \\\hline
  Excepciones                        & \multicolumn{2}{p{10.25cm}}{Cuenta de Outlook inexistente.}
\\\hline
}



\tablaSmallSinColores{Caso de uso 1: Consultar eventos de aulas.}{p{3cm} p{.75cm} p{9.5cm}}{tablaUC1}{
  \multicolumn{3}{l}{Caso de uso 1: Consultar eventos de aulas.} \\
 }
 {
  Descripción                            & \multicolumn{2}{p{10.25cm}}{Permite al usuario visualizar los eventos creados en las aulas, facilitando filtros para una búsqueda mas concreta.} \\\hline
  \multirow{1}{3.5cm}{Requisitos}   &\multicolumn{2}{p{10.25cm}}{RF-6} 
                                         \\\hline
  Precondiciones                         &  \multicolumn{2}{p{10.25cm}}{Haber iniciado sesión.}   \\\hline
    \multirow{2}{3.5cm}{Secuencia normal}  & Paso & Acción \\\cline{2-3}
                                         & 1    & El usuario pincha el botón 'Consultar ocupación de aulas' de la barra superior de navegación.
  \\\cline{2-3}
                                         & 2    & Se redirige a una página donde se muestra un desplegable para elegir el edificio.
  \\\cline{2-3}
                                         & 3    & Se elige el edifico y se muestra un desplegable con las aulas de éste.También se facilita un botón 'Filtros'.
    \\\cline{2-3}
                                         & 4    & Si se elige la opción filtros, se muestran más campos para rellenar y filtrar las aulas por ellos. Si no se puede pinchar el botón 'Mostrar eventos' para ver los eventos del aula elegida en el desplegable.
     \\\cline{2-3}
                                        & 5     & Se muestra la tabla con los eventos pasados y los futuros, especificando detalles sobre ellos.
                                        \\\hline

                                        
  Postcondiciones                        & \multicolumn{2}{p{10.25cm}}{Permite la visualización de los eventos según el aula elegido, depende de los privilegios del usuario dispondrá de las opciones de modificar (CU1.1) y borrar estos eventos (CU1.2).} \\\hline
  Excepciones                        & \multicolumn{2}{p{10.25cm}}{Activar la opción de los filtros pero no rellenarlos todos.}
\\\hline
}


\tablaSmallSinColores{Caso de uso 1.1: Modificar eventos.}{p{3cm} p{.75cm} p{9.5cm}}{tablaUC1.1}{
  \multicolumn{3}{l}{Caso de uso 1.1: Modificar eventos.} \\
 }
 {
  Descripción                            & \multicolumn{2}{p{10.25cm}}{Permite al usuario modificar los eventos existentes.} \\\hline
  \multirow{1}{3.5cm}{Requisitos}   &\multicolumn{2}{p{10.25cm}}{RF-6} 
  \\\cline{2-3}
        &       \multicolumn{2}{p{10.25cm}}{RF-6.1}
                                         \\\hline
  Precondiciones                         &  \multicolumn{2}{p{10.25cm}}{Tener permisos de administrador o de propietario.}   \\\hline
    \multirow{2}{3.5cm}{Secuencia normal}  & Paso & Acción \\\cline{2-3}
                                         & 1    & El usuario pincha el botón 'Consultar ocupación de aulas' de la barra superior de navegación.
  \\\cline{2-3}
                                         & 2    & Se redirige a una página donde se muestra un desplegable para elegir el edificio.
  \\\cline{2-3}
                                         & 3    & Se elige el edifico y se muestra un desplegable con las aulas de éste.También se facilita un botón 'Filtros'.
    \\\cline{2-3}
                                         & 4    & Si se elige la opción filtros, se muestran más campos para rellenar y filtrar las aulas por ellos. Si no se puede pinchar el botón 'Mostrar eventos' para ver los eventos del aula elegida en el desplegable.
     \\\cline{2-3}
                                        & 5     & Se muestra la tabla con los eventos pasados y los futuros, especificando detalles sobre ellos.
    \\\cline{2-3}
                                        & 6     & Se ve un botón de 'Modificar' al lado de los eventos sobre los que tenga permiso para modificar.
    \\\cline{2-3}
                                        & 7     & Se pincha en el botón 'Modificar' y te abre un formulario en la parte superior para cambiar los datos del evento.
                                        \\\hline

                                        
  Postcondiciones                        & \multicolumn{2}{p{10.25cm}}{Si se modifica correctamente, se vuelve a la página anterior. Si hay un error se muestra un mensaje y se permite volver a modificar.} \\\hline
  Excepciones                        & \multicolumn{2}{p{10.25cm}}{Cambiar a una fecha ya reservada.}
  \\\cline{2-3} & \multicolumn{2}{p{10.25cm}}{No tener permisos.}
\\\hline
}




\tablaSmallSinColores{Caso de uso 1.2: Borrar eventos.}{p{3cm} p{.75cm} p{9.5cm}}{tablaUC1.2}{
  \multicolumn{3}{l}{Caso de uso 1.2: Borrar eventos.} \\
 }
 {
  Descripción                            & \multicolumn{2}{p{10.25cm}}{Permite al usuario borrar los eventos existentes.} \\\hline
  \multirow{1}{3.5cm}{Requisitos}   &\multicolumn{2}{p{10.25cm}}{RF-6} 
  \\\cline{2-3}
        &       \multicolumn{2}{p{10.25cm}}{RF-6.2}
                                         \\\hline
  Precondiciones                         &  \multicolumn{2}{p{10.25cm}}{Tener permisos de administrador o de propietario.}   \\\hline
    \multirow{2}{3.5cm}{Secuencia normal}  & Paso & Acción \\\cline{2-3}
                                         & 1    & El usuario pincha el botón 'Consultar ocupación de aulas' de la barra superior de navegación.
  \\\cline{2-3}
                                         & 2    & Se redirige a una página donde se muestra un desplegable para elegir el edificio.
  \\\cline{2-3}
                                         & 3    & Se elige el edifico y se muestra un desplegable con las aulas de éste.También se facilita un botón 'Filtros'.
    \\\cline{2-3}
                                         & 4    & Si se elige la opción filtros, se muestran más campos para rellenar y filtrar las aulas por ellos. Si no se puede pinchar el botón 'Mostrar eventos' para ver los eventos del aula elegida en el desplegable.
     \\\cline{2-3}
                                        & 5     & Se muestra la tabla con los eventos pasados y los futuros, especificando detalles sobre ellos.
    \\\cline{2-3}
                                        & 6     & Se ve un botón de 'Borrar' al lado de los eventos sobre los que tenga permiso para borrar.
    \\\cline{2-3}
                                        & 7     & Se pincha en el botón 'Borrar' y te pide la confirmación del usuario para borrar el evento.
                                        \\\hline

                                        
  Postcondiciones                        & \multicolumn{2}{p{10.25cm}}{Si se borra correctamente, se muestra un mensaje por pantalla y se vuelve a la página anterior. Si se cancela el borrado te mantienes en la misma.} \\\hline
  Excepciones                        & \multicolumn{2}{p{10.25cm}}{No tener permisos.}
\\\hline
}


\tablaSmallSinColores{Caso de uso 2: Consultar información sobre las aulas.}{p{3cm} p{.75cm} p{9.5cm}}{tablaUC2}{
  \multicolumn{3}{l}{Caso de uso 2: Consultar información sobre las aulas.} \\
 }
 {
  Descripción                            & \multicolumn{2}{p{10.25cm}}{Permite al usuario conocer la información relevante acerca de las aulas existentes, sea o no propietario de ellas. La información que se facilita es: nombre, edificio, tipo de aula, capacidad, nº de ordenadores y propietario.} \\\hline
  \multirow{1}{3.5cm}{Requisitos}   &\multicolumn{2}{p{10.25cm}}{RF-3}
                                    \\\cline{2-3}
                                    & \multicolumn{2}{p{10.25cm}}{RF-3.1} \\\cline{2-3}
                                    & \multicolumn{2}{p{10.25cm}}{RF-3.2}
                                         \\\hline
  Precondiciones                         &  \multicolumn{2}{p{10.25cm}}{Haber iniciado sesión.}   \\\hline
    \multirow{2}{3.5cm}{Secuencia normal}  & Paso & Acción \\\cline{2-3}
                                         & 1    & El usuario pincha el botón 'Ver/Modificar aulas' de la barra superior de navegación.
  \\\cline{2-3}
                                         & 2    & Se redirige a la página de información sobre las aulas, donde se muestra una opción para elegir el grupo de aulas.
  \\\cline{2-3}
                                         & 3    & Una vez se elige el grupo de aulas, se pincha en el botón 'Ver aulas'.
    \\\cline{2-3}
                                         & 4    & Se muestra una tabla con las aulas y su información del edificio elegido en el paso anterior.

                                        \\\hline

                                        
  Postcondiciones                        & \multicolumn{2}{p{10.25cm}}{Permite la visualización de las características de las aulas, depende de los privilegios del usuario dispondrá de las opciones de modificar (CU2.1) y borrar (CU2.2) estas aulas.} \\\hline
  Excepciones                        & \multicolumn{2}{p{10.25cm}}{No haber iniciado sesión.}
\\\hline
}


\tablaSmallSinColores{Caso de uso 2.1: Modificar aulas.}{p{3cm} p{.75cm} p{9.5cm}}{tablaUC2.1}{
  \multicolumn{3}{l}{Caso de uso 2.1: Modificar aulas.} \\
 }
 {
  Descripción                            & \multicolumn{2}{p{10.25cm}}{Permite al usuario modificar la información relevante acerca de las aulas existentes. } \\\hline
  \multirow{1}{3.5cm}{Requisitos}   &\multicolumn{2}{p{10.25cm}}{RF-4}
                                    \\\cline{2-3}
                                    & \multicolumn{2}{p{10.25cm}}{RF-4.1} 
                                         \\\hline
  Precondiciones                         &  \multicolumn{2}{p{10.25cm}}{Tener permisos de administrador.}   \\\hline
    \multirow{2}{3.5cm}{Secuencia normal}  & Paso & Acción \\\cline{2-3}
                                         & 1    & El usuario pincha el botón 'Ver/Modificar aulas' de la barra superior de navegación.
  \\\cline{2-3}
                                         & 2    & Se redirige a la página de información sobre las aulas, donde se muestra una opción para elegir el grupo de aulas.
  \\\cline{2-3}
                                         & 3    & Una vez se elige el grupo de aulas, se pincha en el botón 'Ver aulas'.
    \\\cline{2-3}
                                         & 4    & Se muestra una tabla con las aulas y su información del edificio elegido en el paso anterior.
    \\\cline{2-3}
                                         & 5    & Se ve un botón de 'Editar' al lado de la información de las aulas que pueda modificar.
   \\\cline{2-3}
                                         & 6    & Al pinchar sobre éste botón se abre un cuadro en la parte superior de la pantalla donde se pueden introducir los valores a modificar.

                                        \\\hline

                                        
  Postcondiciones                        & \multicolumn{2}{p{10.25cm}}{Si se cambia el propietario, se mostrará un mensaje recordando el cambio manual de propietario que tiene que hacer el administrador.}
  \\\cline{2-3} &   \multicolumn{2}{p{10.25cm}}{Si se cambia correctamente, se vuelve a la pantalla anterior y se muestra un mensaje por pantalla.}
  
  \\\hline
  Excepciones                        & \multicolumn{2}{p{10.25cm}}{No tener permisos.}
  \\\cline{2-3} &  \multicolumn{2}{p{10.25cm}}{Dejar sin rellenar partes del formulario de modificación.}
\\\hline
}


\tablaSmallSinColores{Caso de uso 2.2: Borrar aulas.}{p{3cm} p{.75cm} p{9.5cm}}{tablaUC2.2}{
  \multicolumn{3}{l}{Caso de uso 2.2: Borrar aulas.} \\
 }
 {
  Descripción                            & \multicolumn{2}{p{10.25cm}}{Permite al usuario borrar aulas existentes. } \\\hline
  \multirow{1}{3.5cm}{Requisitos}   &\multicolumn{2}{p{10.25cm}}{RF-4.2}
                                    
                                         \\\hline
  Precondiciones                         &  \multicolumn{2}{p{10.25cm}}{Tener permisos de administrador.}   \\\hline
    \multirow{2}{3.5cm}{Secuencia normal}  & Paso & Acción \\\cline{2-3}
                                         & 1    & El usuario pincha el botón 'Ver/Modificar aulas' de la barra superior de navegación.
  \\\cline{2-3}
                                         & 2    & Se redirige a la página de información sobre las aulas, donde se muestra una opción para elegir el grupo de aulas.
  \\\cline{2-3}
                                         & 3    & Una vez se elige el grupo de aulas, se pincha en el botón 'Ver aulas'.
    \\\cline{2-3}
                                         & 4    & Se muestra una tabla con las aulas y su información del edificio elegido en el paso anterior.
    \\\cline{2-3}
                                         & 5    & Se ve un botón de 'Borrar' al lado de la información de las aulas que pueda eliminar.
   \\\cline{2-3}
                                         & 6    & Al pinchar sobre éste botón se abre un cuadro pidiendo confirmación del usuario para borrar el aula. Ofreciendo una comprobación en la que se permite borrar definitivamente o cancelar el borrado.

                                        \\\hline

                                        
  Postcondiciones                        & \multicolumn{2}{p{10.25cm}}{Si se borra el aula,se muestra un mensaje por pantalla avisando del borrado y se vuelve al formulario anterior.}
  
  \\\hline
  Excepciones                        & \multicolumn{2}{p{10.25cm}}{No tener permisos.}
\\\hline
}



\tablaSmallSinColores{Caso de uso 3: Realizar reservas.}{p{3cm} p{.75cm} p{9.5cm}}{tablaUC3}{
  \multicolumn{3}{l}{Caso de uso 3: Realizar reservas.} \\
 }
 {
  Descripción                            & \multicolumn{2}{p{10.25cm}}{Permite al usuario realizar reservas sobre las aulas que tenga permisos siempre que la hora no esté ocupada.} \\\hline
  \multirow{1}{3.5cm}{Requisitos}   &\multicolumn{2}{p{10.25cm}}{RF-5}
                                    \\\cline{2-3}
                                    & \multicolumn{2}{p{10.25cm}}{RF-5.1} \\\cline{2-3}
                                    & \multicolumn{2}{p{10.25cm}}{RF-5.2}
                                    \\\cline{2-3}
                                    & \multicolumn{2}{p{10.25cm}}{RF-5.3}
                                         \\\hline
  Precondiciones                         &  \multicolumn{2}{p{10.25cm}}{Haber iniciado sesión. Tener permisos necesarios.}   \\\hline
    \multirow{2}{3.5cm}{Secuencia normal}  & Paso & Acción \\\cline{2-3}
                                         & 1    & El usuario (administrador o propietario) pincha el botón 'Realizar reserva' de la barra superior de navegación.
  \\\cline{2-3}
                                         & 2    & Se muestra un pequeño formulario en el que se piden la capacidad, nº de ordenadores, edificio y tipo de aula.
  \\\cline{2-3}
                                         & 3    & Si se pincha en el botón 'Consultar'. Se muestran las 3 formas distintas en que se puede realizar la reserva. Se selecciona una para seguir.
\\\cline{2-3}
                                         & 4    & Se pincha en el botón 'Aplicar'. Esto muestra un formulario para rellenar la información requerida según el tipo de reserva seleccionada previamente.
\\\cline{2-3}
                                         & 5    & Si se pincha en el botón 'Consultar ocupación de aulas', se muestran las horas reservadas del aula elegida, si la elección es 'Cualquiera', se muestra de todas las aulas.
\\\cline{2-3}
                                         & 6    & Si se pincha en el botón 'Crear evento' para reservar la hora y aula pedida si están libres, sino se mostrará un error por pantalla.

                                        \\\hline

                                        
  Postcondiciones                        & \multicolumn{2}{p{10.25cm}}{Se permite volver a reservar otro aula recordando la reserva recién realizada por si se quiere una mínima modificación.} \\\hline
  Excepciones                        & \multicolumn{2}{p{10.25cm}}{No existan aulas con los requisitos específicos.}
\\\hline
}




\tablaSmallSinColores{Caso de uso 4: Crear aulas.}{p{3cm} p{.75cm} p{9.5cm}}{tablaUC4}{
  \multicolumn{3}{l}{Caso de uso 4: Crear aulas.} \\
 }
 {
  Descripción                            & \multicolumn{2}{p{10.25cm}}{Permite al usuario administrador crear aulas nuevas.} \\\hline
  \multirow{1}{3.5cm}{Requisitos}   &\multicolumn{2}{p{10.25cm}}{RF-7}
                                    \\\cline{2-3}
                                         \\\hline
Precondiciones                         &  \multicolumn{2}{p{10.25cm}}{Haber iniciado sesión. Ser administrador.}   \\\hline
    \multirow{2}{3.5cm}{Secuencia normal}  & Paso & Acción \\\cline{2-3}
                                         & 1    & El usuario (administrador) pincha el botón 'Añadir aulas' de la barra superior de navegación.
  \\\cline{2-3}
                                         & 2    & Se muestra un formulario en el que se puede seleccionar el edificio donde se ubica el aula, el tipo de aula y el propietario. Se dejan campos a rellenar para proporcionar el nombre del aula, la capacidad y el nº de ordenadores.
  \\\cline{2-3}
                                         & 3    & Se pincha en el botón 'Crear aula'.


                                        \\\hline
 Postcondiciones                        & \multicolumn{2}{p{10.25cm}}{Asignar manualmente los permisos desde Outlook al propietario del nuevo aula.} \\\hline
  Excepciones                        & \multicolumn{2}{p{10.25cm}}{No ser administrador.}
  \\\cline{2-3}
                        &\multicolumn{2}{p{10.25cm}}    {No llenar todos los campos requeridos}
                        \\\hline

}
                                        