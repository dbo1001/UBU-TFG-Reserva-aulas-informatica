\capitulo{6}{Trabajos relacionados}

Esta aplicación web tiene similitudes con todas las aplicaciones de reserva de espacios, pero en lo que se distingue esta aplicación principalmente es en el uso de Outlook tanto para la validación del usuario como para la gestión de los calendarios usando Outlook calendar.\newline
En cuanto a funcionalidades, mirando aplicaciones similares de reservas, la principal diferencia a destacar es que ésta está realizada para la universidad o entidades similares, ya que se tiene en cuenta distintos edificios en los que se manejan distintas aulas y dichas aulas tienen características y se proporcionan filtros concretos de aulas de universidad.\newline Al hacer una búsqueda de aplicaciones que guardan cierta  similitud con la presente, se han encontrado las siguientes:
\begin{itemize}
    \item \textit{BookedApp}\newline 
    Es una aplicación que se dedica a la reserva de sitios individuales, no de aulas, sino de bibliotecas, a diferencia de la aplicación desarrollada en el presente trabajo que gestiona más de un espacio pero no por sitios individualmente, sino reservas de aulas completas. \cite{BookedApp}
    \item  \textit{Booked Scheduler – Reserva de aulas}\href{https://valentingom.wordpress.com/2016/07/10/booked-scheduler-reservas/}{Enlace}\cite{BookedScheduler} \newline
    Es una aplicación más desarrollada que permite reservas de aulas de forma similar a la del presente trabajo, pero no está sincronizado con Outlook y todo se gestiona a través de una base de datos, sin embargo como ventaja suya podemos apreciar los eventos representados en instancias de calendarios.
\end{itemize}


