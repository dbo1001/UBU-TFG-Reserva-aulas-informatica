\capitulo{4}{Técnicas y herramientas}

\section{Técnicas}

\subsection{\textit{Scrum}}
La metodología \textit{Scrum} es una metodología de desarrollo ágil, en la que se aplican buenas prácticas y procesos con la finalidad de obtener la mejor calidad posible en el producto final buscado \cite{scrum}.\newline
Esta metodología está pensada para un equipo de trabajo, con el fin de organizarse y trabajar más eficientemente obteniendo los mejores resultados posibles. En el presente trabajo se trató de simular una metodología \textit{Scrum} ya que el trabajo lo realiza un único alumno, pero la intención es dividir dicho trabajo en \textit{sprints} que contendrán dentro las tareas a realizar.\newline
Esta metodología se basa en crear \textit{sprints}, dentro de los que se crearán las tareas a realizar (\textit{issues}). Para establecer estas tareas dentro de cada sprint, se realiza una planificación, en la que se deja claro los objetivos a conseguir y el tiempo para cada uno de ellos. 
Se realizan revisiones para controlar los avances de los sprint planificados y llevar un control sobre el proyecto.



\section{Herramientas}
\subsubsection{\textit{GitHub}}
Para el control de versiones de este proyecto he utilizado \textit{GitHub}, que es un repositorio en línea que emplea \textit{Git}. Así podemos guardar los cambios que vamos realizando en el proyecto y controlar que se cumplan las tareas en el tiempo indicado. Para llevar un mejor control de las tareas activas y lo pendiente se utilizó ZenHub, que facilita un panel en el que se cambian los estados de dichas tareas. 

\subsection{GitTortoise}
Es el cliente de control de revisiones que he utilizado para subir los cambios y tener actualizado el GitHub. Permite comparar versiones y solucionar conflictos entre ellas, facilita la subida de código.\cite{wiki:gitTortoise}
\subsection{\textit{Visual Studio Code}}
Editor y compilador de código fuente. Incluye el soporte que ofrece para la depuración y para el control integrado de \textit{Azure Web Services} el cual ha sido utilizado en el presente trabajo.



\subsection{JavaScript}
Es un lenguaje de programación ligero e interpretado, orientado a objetos pero basado en prototipos. En mi caso utilizado como lenguaje de scripting para páginas web. Se ha utilizado desde el lado del cliente proporcionando mejoras de las interfaz y de la comunicación con el servidor \cite{wiki:javascript}.



\subsection{LaTeX}
Como herramienta para realizar la documentación se ha escogido \textit{LaTeX}, está diseñado para crear documentos con una alta calidad tipográfica. Como editor de \textit{LaTeX} he utilizado \texit{Overleaf}

\subsection{Azure}
Es un servicio de computación en la nube, de Microsoft, diseñado para construir, probar, desplegar y administrar aplicaciones y servicios mediante el uso de sus centros de datos. Proporciona muchas herramientas para facilitar el manejo de estas aplicaciones o servicios. En el caso de este trabajo se ha utilizado para desplegar la aplicación web y la base de datos. Ambas pueden ser gestionadas desde extensiones del Visual Studio Code, el programa que utilizo para el desarrollo del proyecto. \cite{wiki:azure}.\newline
También ha sido necesario utilizar el servicio de administración de identidades de Azure, conocido como Azure Active Directory, para registrar la aplicación y poder proporcionar un inicio de sesión único a ésta.

\subsection{Outlook API REST (Microsoft Graph) }
Para realizar este proyecto se ha utilizado la \texit{API REST} de \texit{Outlook} para manejar los calendarios, que equivalen a las aulas que se proporcionan al usuario para reservar. Para tener acceso a las herramientas de esta API REST, se usó el punto de conexión Microsoft Graph ya que ofrece más servicios y características siguiendo la recomendación de Microsoft (\url{https://docs.microsoft.com/es-ES/outlook/rest/compare-graph} \cite{microsoftGraph}).
En concreto para este proyecto he utilizado distintas funciones de esta \texit{API REST}:
\begin{itemize}
    \item \texbf{Calendarios}: Para realizar modificaciones u obtener información de los calendarios del usuario.
    \item \texbf{Correo}: Para enviar los mensajes automáticos.
    \item \texbf{Grupos de calendarios}: Para obtener los grupos de calendarios (equivalen a los edificios) y la información que contiene cada uno de ellos.
\end{itemize}
Una vez tenemos la cuenta iniciada recibimos los tokens de autenticación y de refresco, como ya vimos anteriormente, estos se utilizan para validar la sesión si se caduca el token de acceso o recoger uno nuevo al iniciar una sesión distinta, para ello se envía a la dirección proporcionada por Microsoft al registrar la app.\newline
Dentro de este punto de conexión a la API REST hemos utilizado varias llamadas distintas, las más destacables y usadas en el proyecto son relacionadas a las llamadas de los calendarios, estas son las siguientes:
\begin{itemize}
    \item Obtener el id de los calendarios del usuario que esté logeado o del grupo de calendario especificado.
    \item Añadir eventos nuevos al calendario especificado mediante el id obtenido en el punto anterior.
    \item Obtener los grupos de calendarios (que en este proyecto los conocemos como edificios).
    \item Obtener los id de los eventos de cada calendario para poder modificarlos o borrarlos.
\end{itemize}



\subsection{Pylint}
Pylint es una herramienta de análisis de código estático de Python que utilizo en Visual Studio Code, busca errores de programación, ayuda a aplicar un estándar de codificación y ofrece sugerencias de refactorización simples. Es un software altamente configurable y gratuito.\cite{pylint}

\subsection{SQL Server}
Es un sistema de gestión de base de datos relacional, desarrollado por la empresa Microsoft, utilizado por Azure para gestionar los servicios de bases de datos en la nube. El entorno de trabajo que se ha utilizado ha sido la propia página de Azure, que permite la creación de tablas y ejecución de consultas, ya que primeramente fue creado en MySQL y solamente había que trasladarlo.\cite{wiki:sqlServer}

\subsection{JWT} 
JWT (JSON Web Token) es un estándar en el que se define un mecanismo para poder propagar entre dos partes, y de forma segura, la identidad de un determinado usuario. Se trata de una cadena de texto que tiene tres partes codificadas en Base64, cada una de ellas separadas por un punto. En este proyecto, necesitamos una función que pueda decodificar el JWT del id recibido al iniciar sesión, para obtener la información necesaria como token de acceso, token para refrescarlo o email de la sesión.\cite{JWT}

\subsection{Herramientas web}

\subsubsection{Flask}
Flask es un 'micro' framework en lenguaje Python que permite y facilita la creación de aplicaciones web con un comienzo sencillo y una escalabilidad muy alta. Utiliza la especificación Werkzeug y Jinja\cite{wiki:jinja2} como motor en los templates.\newline
Que sea un framework de tipo 'micro' significa que al instalar Flask tenemos las herramientas necesarias para crear una aplicación web funcional, donde se mantiene el núcleo simple pero extensible, a diferencia de frameworks full stack que proporcionan herramientas y módulos internos para soportar aplicaciones más grandes \cite{FlaskDefinicion}.

\subsubsection{\textit{Jinja2}}
\textit{Jinja2} es un motor de plantillas web para aplicaciones desarrolladas en \textit{Python}. Permite que \textit{Flask} pueda hacer uso de los contenidos de las plantillas \textit{HTML} \cite{wiki:jinja2}.

\subsubsection{Bootstrap}
Es una biblioteca de código abierto que proporciona estilos, plantillas y diseños ya creados para usar directamente en nuestras aplicaciones web \cite{wiki:bootstrap}.

\subsubsection{Werkzeug}
Werkzeug es una biblioteca de aplicaciones web de WSGI (interfaz de puerta de enlace del servidor web), se utiliza internamente en Flask y entre otras muchas, realiza las siguientes funciones:
\begin{itemize}
\item Depurador para inspeccionar rastros de la pila y el código fuente.
\item Objeto de solicitud para interactuar con encabezados, argumentos de consulta, datos de formulario, archivos y cookies.
\item Objeto de respuesta en la transmisión de datos.
\item Enrutamiento de URL.
\item Utilidades HTTP.
\end{itemize}